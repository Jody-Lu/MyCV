%%%%%%%%%%%%%%%%%%%%%%%%%%%%%%%%%%%%%%%%%
% Medium Length Professional CV
% LaTeX Template
% Version 2.0 (8/5/13)
%
% This template has been downloaded from:
% http://www.LaTeXTemplates.com
%
% Original author:
% Trey Hunner (http://www.treyhunner.com/)
%
% Important note:
% This template requires the resume.cls file to be in the same directory as the
% .tex file. The resume.cls file provides the resume style used for structuring the
% document.
%
%%%%%%%%%%%%%%%%%%%%%%%%%%%%%%%%%%%%%%%%%

%----------------------------------------------------------------------------------------
%	PACKAGES AND OTHER DOCUMENT CONFIGURATIONS
%----------------------------------------------------------------------------------------

\documentclass[letterpaper]{resume} % Use the custom resume.cls style

\usepackage[left=0.75in,top=0.6in,right=0.75in,bottom=0.6in]{geometry} % Document margins
\usepackage{fontspec}
\setmainfont{Times New Roman}

\name{Meng-Ju Leu} % Your name
\address{b00901192@ntu.edu.tw}
\address{+886-985-292-418} % Your phone number and email
\address{No.8, Aly.46, Ln, Zhongzhen Rd., Shalu Dist, Taichung, Taiwan} % Your address

\begin{document}

%----------------------------------------------------------------------------------------
%	OBJECTIVE SECTION
%----------------------------------------------------------------------------------------

%\begin{rSection}{Objective}
%To obtain a full time Junior Software Engineer position where I could contribute my technical skill in system-level / computer network / cloud computing area and communication strength for teams interaction
%\end{rSection}
%\vspace{-0.6\baselineskip}
%----------------------------------------------------------------------------------------
%	EDUCATION SECTION
%----------------------------------------------------------------------------------------

\begin{rSection}{Education}
{\bf National Taiwan University}, Taipei, Taiwan \hfill {09-2012 -- 07-2016} \\ 
Bachelor of Science in Electrical Engineering
\end{rSection}
\vspace{-0.6\baselineskip}


%----------------------------------------------------------------------------------------
%	PROFSESSIONAL EXPERIENCE SECTION
%----------------------------------------------------------------------------------------

\begin{rSection}{Professional Experience}
\begin{rSubsection}{KKBOX Inc.}{01-2016 - Present}{Intern Software Quality Assurance Engineer}{Taipei, Taiwan}
\item Extended the open-source project Pyuia to accommodate perfectly on Appium v1.5.3
\item Built KKBOX Android Apps UI-Test automation framework based on Selenium and Robot Framework and maintained the whole testing library.
\item Packed auto-testing environment into docker image, so that we can start our test plan without tedious environment setting.
\item Built api-testing framework for testing KKBOX database using nosetest 
\end{rSubsection}
\vspace{-0.6\baselineskip}
%------------------------------------------------

\begin{rSubsection}{Chunghwa Telecom Co. Ltd.  \textnormal{}}{01-2015 -- 01-2016}{Research Assistant}{Taipei, Taiwan}
\item Designed algorithms to collect network data for reliable and accurate representation for pricing schemes.
\item Emulated SDN topology with commercial software -- EstiNet and wrote Python program to collect network flow data through Ryu, an open source SDN-controller.
\item Proposed novel dynamic data pricing scheme, especially time-dependent pricing and location-based data pricing to solve network congestion.
\end{rSubsection}

\end{rSection}
\vspace{-0.6\baselineskip}

%----------------------------------------------------------------------------------------
%	RESEARCH & PROJECT SECTION
%----------------------------------------------------------------------------------------

\begin{rSection}{Research \& Course Project}

\begin{rSubsection}{FRAIG: Functionally Reduced And Inverter Graph}{01-2015}{}{}
\item Developed C++ programs, optimized a circuit by removing unused gate, performed trivial optimization and simplification by structural hash and equivalent gate merging.
\end{rSubsection}

\vspace{-0.6\baselineskip}
\begin{rSubsection}{Machine Learning Course Project}{Fall Session 2015}{}{}
\item Implemented machine learning algorithms including soft margin SVM, AdaBoost-Stump algorithm, Decision Tree, PLA and Decision Stump algorithm, Logistic Regression and Regularized Linear Regression with Python.
\end{rSubsection}
\vspace{-0.6\baselineskip}
\begin{rSubsection}{Advance Performance Modeling on PCS Networks}{06-2015}{}{}
\item Built a network simulation model and implemented it with C. Controlled the error under 2 percent from the real world data.
\end{rSubsection}

\end{rSection}
\vspace{-0.6\baselineskip}


%----------------------------------------------------------------------------------------
%	Publication
%----------------------------------------------------------------------------------------

\begin{rSection}{Publications}

\begin{rSubsection}{Location-Based Time-Dependent Smart Data Pricing by SDN}{}{}{}
\item IEEE GLOBECOM 2016 in Washington D.C.
\end{rSubsection}

\end{rSection}
\vspace{-0.6\baselineskip}
%----------------------------------------------------------------------------------------
%	TECHNICAL SKILLS SECTION
%----------------------------------------------------------------------------------------

\begin{rSection}{Technical Skills}

\begin{rSubsection}{}{}{}{}
\vspace{-0.6\baselineskip}
\item {\bfseries{Programming Language:}} Python / Matlab / C / C++ / Swift / Verilog
\item {\bfseries{CS Fundamental:}} Data Structure and Algorithm / Operating System / Computer Architecture
\item {\bfseries{Debug Tools / Version Control:}} GDB / Github 
\item {\bfseries{Language:}} Chinese and English (Fluent)
\item {\bfseries{EE Experience:}} DSP / Analog and Digital communication systems / Circuit Design
\end{rSubsection}

\end{rSection}
\vspace{-0.6\baselineskip}
%----------------------------------------------------------------------------------------
%	OTHERS
%----------------------------------------------------------------------------------------

%----------------------------------------------------------------------------------------

\end{document}
